\documentclass{article}
\title{Rapport d'OS}
\author{Andrius Ezerskis \& Moïra Vanderslagmolen \& Enes}
\begin{document}

\maketitle{Introduction}

\maketitle{Stockage de la base de données}

Afin de stocker les données de milliers d'étudiants, plusieurs choix se sont offerts à nous. 
Tout d'abord, nous avons utilisé std::map, comparable à un dictionnaire en python, dont les
clés ont été l'identifiant de l'élève et les valeurs la struct student. Seulement, cela
prenait beaucoup de place en mémoire et nous avons donc délaissé l'idée. Nous avons ensuite essayé
la classe vector.

\maketitle{Processus vs Threads}
Nous avons choisi des threads car ils ont une mémoire partagée, contrairement aux processus
\section{ahh}
\end{document}
