\documentclass{article}
\title{Rapport d'OS}
\author{Andrius Ezerskis \& Moïra Vanderslagmolen \& Enes}
\begin{document}

\maketitle{Introduction}

\section{Stockage de la base de données}

Afin de stocker les données de milliers d'étudiants, plusieurs choix se sont offerts à nous. 
Tout d'abord, nous avons utilisé std::map, comparable à un dictionnaire en python, dont les
clés ont été l'identifiant de l'élève et les valeurs la struct student. Seulement, cela
prenait beaucoup de place en mémoire et nous avons donc délaissé l'idée. Nous avons ensuite essayé
la classe vector.

\section{Processus vs Threads}
Nous avons choisi des threads car ils ont une mémoire partagée, contrairement aux processus
\section{Améliorations possibles}
    \subsection{std::map}
    Nous avions comme idée de stocker, dans un structure de données std::map (équivalent d'un dictionnaire en Python), les identifiants des étudiants en clés ainsi que leur indice de position dans la base de données en valeurs de ces clés. \newline
    Selon nous, cela aurait permis de faciliter et d'améliorer l'efficacité de la vérification de si un étudiant est déjà dans la base de données ou non. \newline En effet, lors de l'ajout d'un étudiant, nous aurions juste à vérifier si l'identifiant est présent dans le map pour éviter d'avoir des doublons dans la base de données.

\end{document}
