
\documentclass[utf8]{article}
\usepackage[utf8]{inputenc}

\usepackage[parfill]{parskip}
\usepackage{booktabs}
\usepackage{amsmath}
\usepackage{amssymb}
\usepackage{amsfonts}
\usepackage{graphicx}
\usepackage{float}
\usepackage{listingsutf8}
\usepackage{listings}
\usepackage{graphicx}
\usepackage{fullpage}
\title{Rapport d'OS}
\author{Andrius Ezerskis \& Moïra Vanderslagmolen \& Hasan Yildirim}
\begin{document}

\section{Introduction}

\section{Stockage de la base de données}

Afin de stocker les données de milliers d'étudiants, plusieurs choix se sont offerts à nous. 
Tout d'abord, nous avons utilisé std::map, comparable à un dictionnaire en python, dont les
clés ont été l'identifiant de l'élève et les valeurs la struct student. Seulement, cela
prenait beaucoup de place en mémoire et nous avons donc délaissé l'idée. Nous avons 

\subsection{Checker l'identifiant}

Pour voir si l'identifiant est déjà présent dans la base de données, nous avons pensé à mettre chaque étudiant
dans une array (étudiant id=500 va à l'emplacement db[500]). La recherche par id et le delete auraient été rapides.
Seulement, si l'utilisateur introduit un id très grand, alors le stockage de la data base sera très grand, pour très peu d'étudiants.
Nous avons donc décidé de partir sur une recherche dichotomique et de trier les étudiants par id(utilisation de la librairie binarysearch)

\section{Fin d'une query}

Pour signaler la fin d'une query, nous avons implémenté une message queue grâce à une pipe. Lorsqu'on envoie 
une query à un processus, nous incrémentons une variable globale OperationInProgress qui indique le nombre d'opérations en cours.
Lorsque qu'un processus termine sa query, il écrit dans une pipe. Le processus principal va lire cette pipe et va décrémenter
la variable globale au fur et à mesure. Le pipe est non-bloquant, ce qui permet au processus principal de continuer
même si il ne lit rien dans la pipe. Lors d'une transaction, on limite OperationInProgress à 1.

\section{Gestion de signaux}

Plusieurs choix se sont offerts à nous afin de terminer un programme. Nous avons décidé de garder la message queue 
et d'attendre que la message queue soit terminée, et d'envoyer un signal KILL, qui arrête la boucle du programme.


\section{Classes}

Nous avons mis la database\_t et le query\_result\_t dans une classe, afin de rendre le code plus lisible. Par conséquent,
nous avons implémenté un constructeur dans query\_result\_t, au lieu du query\_result\_init, pour que le code
crée directement le query\_result\_t, sans devoir appeller l'initialisation à chaque création d'objet.
Nous n'avons pas mis student\_t en classe, parce que les attributs d'un student sont souvent accédés par la classe query result t.
\section{Améliorations possibles}
    \subsection{std::map}
    Nous avions comme idée de stocker, dans un structure de données std::map (équivalent d'un dictionnaire en Python), les identifiants des étudiants en clés ainsi que leur indice de position dans la base de données en valeurs de ces clés. \newline
    Selon nous, cela aurait permis de faciliter et d'améliorer l'efficacité de la vérification de si un étudiant est déjà dans la base de données ou non. \newline En effet, lors de l'ajout d'un étudiant, nous aurions juste à vérifier si l'identifiant est présent dans le map pour éviter d'avoir des doublons dans la base de données.

\end{document}
